\documentclass[10pt,twocolumn,letterpaper]{article}

\usepackage{cvpr}
\usepackage{times}
\usepackage{epsfig}
\usepackage{graphicx}
\usepackage{amsmath}
\usepackage{amssymb}

% Include other packages here, before hyperref.

% If you comment hyperref and then uncomment it, you should delete
% egpaper.aux before re-running latex.  (Or just hit 'q' on the first latex
% run, let it finish, and you should be clear).
\usepackage[breaklinks=true,bookmarks=false]{hyperref}

\cvprfinalcopy % *** Uncomment this line for the final submission

\def\cvprPaperID{****} % *** Enter the CVPR Paper ID here
\def\httilde{\mbox{\tt\raisebox{-.5ex}{\symbol{126}}}}

% Pages are numbered in submission mode, and unnumbered in camera-ready
%\ifcvprfinal\pagestyle{empty}\fi
\setcounter{page}{1}
\begin{document}

%%%%%%%%% TITLE
\title{Mathematical Expression Recognizer}

\author{Kim, ChanMin \qquad HeeHoon Kim \qquad Sanghyeok Park\\
Department of Computer Science and Engineering\\
Seoul National University\\
{\tt\small kcm1700@snu.ac.kr, csehydrogen@gmail.com, pps987@snu.ac.kr}
% For a paper whose authors are all at the same institution,
% omit the following lines up until the closing ``}''.
% Additional authors and addresses can be added with ``\and'',
% just like the second author.
% To save space, use either the email address or home page, not both
}

\maketitle
%\thispagestyle{empty}

%%%%%%%%% ABSTRACT
\begin{abstract}

Our goal is to write a program which recongnizes handwritten mathematical expressions.
We made several milestones to solve this hard problem.
The problem can be solved in three stages, namely character segmentation, character recognition, and calculation.
We will mainly focus on the latter two stages.
We are going to use artificial neural network algorithm to solve this problem.

\end{abstract}

%%%%%%%%% BODY TEXT
\section{Problem}

Writing mathematical expressions in a computer readable format is difficult and requires a lot of efforts to get used to.
For example, the \LaTeX syntax requires a lot of cumbersome notations. \verb'\frac{3}{5}' for $\frac{3}{5}$, \verb'\sum_{i=0}^{10}{i \times 3}' for $\sum_{i=0}^{10}{i \times 3}$.

We will recognize handwritten mathematical expressions by image processing.
We will also try to calculate the value of the expression after the recognition.



%-------------------------------------------------------------------------
\section{Goal}

Our goal is to make a recognizer for handwritten mathematical expressions. We will also calculate the recognized expressions and show the results if possible.

It is difficult to recognize arbitrary mathematical expressions at first, so we made some milestones of our goal.

\begin{enumerate}
\item The expression contains only digits and some arithmetic operators. \\
e.g.) $765+573-25252$

\item The expression contains parentheses. \\
e.g.) $(301-314)\times 147$

\item The expression contains numbers with a decimal point. \\
e.g.) $147.46 + 147.47 - 4.30 \times 18$

\item The expression contains pi and trigonometrical functions. \\
e.g.) $\sin \pi + \cot (\pi / 4)$

\item The expression contains fractions and some mathematical symbols. \\
e.g.) $$ \frac{3}{5} = 0.6 $$

\item The expression contains sigma(summation) symbols and integration symbols. \\
e.g.) $$ \sum_{i=0}^{5} i \times 2 $$

\end{enumerate}

\section{Previous Works}
`Character segmentation in handwritten words — An overview'
\cite{Lu199677} summarizes various methods used in segmenting hand-written characters.
Some algorithms are based on tracking important contour features.
Others depend on recognition algorithm,
which calculates likelihood score to be isolated symbols,
and repeats merging and separating.

`Neural Networks and Deep Learning' \cite{MichaelNielsen} shows basic technique to recognize hand-written digits using Artificial Neural Network.

There was a similar student project `OCR of Math Expressions'\cite{OCRMATH}.
They used printed mathematical expressions as input.

%------------------------------------------------------------------------
\section{Methods}

The proposed problem might be solved in several stages:
character segmentation, character recognition, and calculating final results.

The first stage `character segmentation' is to detect characters and to output each character region.
This is probably the most challenging part, therefore we will not address the segmentation issue
unless we finish other parts.
We will just assume the segmentation is done already.

For the next stage `character recognition',
we will apply convolutional neural network to recognize each symbol in an expression.
We also have to consider context to get proper expression. For example, $\sin \pi$ is much more likely to appear in the expression than $51n \pi$. Markov chain might be used to determine the probabilities.

Finally, we will calculate the expressions based on the results of the prior stages.


%-------------------------------------------------------------------------
\section{Evaluation}


\subsection{Training Data}

We will use training data of "MNIST database of handwritten digits"
and CROHME (Competition on Recognition of Online Handwritten Mathematical Expressions) dataset. 

\subsection{Test Data}

Each of the above two data sets provides training data and test data separately.
We can use test data from those sets.

\subsection{Evaluation Procedure}

At first, using MNIST database, we can evaluate our recognition ability digit by digit.
Moreover, using CROHME dataset, we can also determine our recongnition ability of mathematical expressions.
We can scale our ability using error rate, or calculating string distances.

%-------------------------------------------------------------------------
\section{Team and Role}

There will be no specific roles assigned. We (Kim, ChanMin, HeeHoon Kim, and Sanghyeok Park) will do all parts together.

%-------------------------------------------------------------------------
\section{References}

{\small
\bibliographystyle{ieee}
\bibliography{egbib}
}

\end{document}
